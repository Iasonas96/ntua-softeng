\documentclass[a4paper,oneside, 12pt]{article}

\usepackage[margin=0.7in]{geometry}
\usepackage[cm-default]{fontspec}
\usepackage{xunicode}
\usepackage{xltxtra}
\usepackage{xgreek}
\usepackage{graphicx}
\usepackage{hyperref}
\usepackage{cleveref}

\graphicspath{{../UML/}{UML/}}

\setmainfont[Mapping=tex-text]{Linux Libertine O}

\title{Stakeholders Requirements Specification\\
\begin{flushleft}
	Καταστηματάρχες
\end{flushleft}}
\date{\vspace{-5ex}}

\begin{document}
\maketitle
\section{Εισαγωγή}
\subsection{Ταυτότητα-επιχειρησιακοί στόχοι}

Οι καταστηματάρχες είναι ιδιοκτήτες πρατηρίων υγρών καυσίμων ανά την επικράτεια.
Χρησιμοποιώντας το παρατηρητήριο τιμών υγρών καυσίμων έχουν τη δυνατότητα να
προβάλουν την επιχείρηση τους μέσα από τη συνεχή ενημέρωση των καταναλωτών για
τις τιμές διάθεσης των καυσίμων. Βελτιώνουν την ανταγωνιστικότητα της επιχείρησης
τους αποκτώντας πρόσβαση σε πληροφορίες για τις τιμές της αγοράς. Τέλος, μπορούν
να μεγιστοποιήσουν την ποιότητα των παρεχόμενων υπηρεσιών προς τους πελάτες τους
μέσω των στατιστικών που τους προσφέρει το παρατηρητήριο.

\subsection{Περίγραμμα επιχειρησιακών λειτουργιών}

Ο πρατηριούχος εγγράφεται στο παρατηρητήριο παρέχοντας τα απαραίτητα
διαπιστευτήρια για την ιδιότητα του. Για να διατηρήσει την εγγραφή του καταβάλει
περιοδικά ένα τέλος συνδρομής. Αποκτά έτσι πρόσβαση στο σύστημα με το ρόλο του
καταστηματάρχη, επιτρέποντας του να παρακολουθεί και να ενημερώνεται για το
ενδιαφέρον των εθελοντών χρηστών για την επιχείρηση του και τις ανταγωνιστικές
επιχειρήσεις και να ελέγχει την εγκυρότητα των καταγραφών τιμής για το πρατήριο
του. Παράλληλα, δεσμεύεται να εξαργυρώνει με τη μορφή εκπτώσεων στο κατάστημα
του τους πόντους που συλλέγουν μέσω του παρατηρητηρίου οι πελάτες του, ως
επιβράβευση για το έργο τους.

\section{Αναφορές-πηγές πληροφοριών}

\begin{itemize}
		
	\item \href{http://www.fuelprices.gr/}{Παρατηρητήριο Τιμών Υγρών Καυσίμων }
	του Υπουργείου Ανάπτυξης και Ανταγωνιστικότητας
	
	\item \href{https://fuelgr.gr}{Εφαρμογή }
	ενημέρωσης για πρατήρια και τιμές πώλησης καυσίμων
	
	
\end{itemize}

\section{Διαχειριστικές απαιτήσεις επιχειρησιακού περιβάλλοντος}

\subsection{Επιχειρησιακό μοντέλο}

Ο πρατηριούχος με την εγγραφή του στο παρατηρητήριο (την οποία διατηρεί
καταβάλλοντας περιοδικά ένα αντίτιμο) αποκτά πρόσβαση σε δεδομένα και στατιστικά
πολύτιμα για την βελτίωση της ανταγωνιστικότητας του, που του παρέχει το
παρατηρητήριο. Αυτά αφορούν συγκεκριμένα το κατάστημα του (όπως το ενδιαφέρον
των χρηστών) αλλά και συνολικά την αγορά. Επίσης, επιβραβεύει με εκπτώσεις τους
εθελοντές που παρέχουν σωστές καταγραφές για το πρατήριο του (όπως διαπιστώνεται
από το σύστημα πόντων του παρατηρητηρίου). Με αυτόν τον τρόπο δίνει κίνητρο για καλύτερη κάλυψη της επιχείρησης του, η οποία συνεπάγεται ευρύτερη προβολή της,
αποσκοπώντας σε αντίστοιχη αύξηση της πελατείας.

\subsection{Περιβάλλον διαχείρισης πληροφοριών}

Σήμερα, η μόνη παρόμοια λύση, για την κεντρική συγκέντρωση και δημόσια διάθεση των
τιμών υγρών καυσίμων στην αγορά λιανικής είναι το \href{http://www.fuelprices.gr/}
{Παρατηρητήριο Τιμών Υγρών Καυσίμων } του Υπουργείου Ανάπτυξης και Ανταγωνιστικότητας.
Κάθε άλλη αντίστοιχη εφαρμογή (όπως το \href{https://fuelgr.gr/}{FuelGr})
λαμβάνει τα δεδομένα από αυτό και απλώς τα επαναπροωθεί. Η σημαντικότερη διαφορά
είναι ότι αυτό λειτουργεί με διαφορετικό μοντέλο, καθώς υπεύθυνοι για την ενημέρωση
των τιμών είναι οι ίδιοι οι καταστηματάρχες και οι χρήστες δεν αποτελούν μέρος του
συστήματος.

\section{Λειτουργικές απαιτήσεις επιχειρησιακού περιβάλλοντος}

\subsection{Επιχειρησιακές διαδικασίες}

\begin{itemize}
	
\item Ο καταστηματάρχης εγγράφεται στο παρατηρητήριο (με τον αντίστοιχο ρόλο)
παρέχοντας διαπιστευτήρια για την ταυτότητα και την ιδιότητα του.
\item Μπορεί να κάνει αναφορά ανακρίβειας για την καταγραφή κάποιας τιμής για την
επιχείρηση του.
\item Εξαργυρώνει (ως έκπτωση) τους πόντους του συστήματος που έχουν συλλέξει οι
εθελοντές από την καταγραφή τιμών για την επιχείρηση του.
\item Λαμβάνει δεδομένα και στατιστικά από το παρατηρητήριο για το κατάστημα του
και συνολικά για την αγορά.

\end{itemize}

\subsection{Περιορισμοί}

\begin{itemize}
	\item Για την εγγραφή στο σύστημα θα πρέπει να παρέχονται τα απαραίτητα
	διαπιστευτήρια τα οποία θα επαληθεύουν την ταυτότητα του καταστηματάρχη.
	\item  Ο καταστηματάρχης δεν θα μπορεί να προσθέσει καταγραφή τιμής για καμία
	επιχείρηση.
	\item  Ο καταστηματάρχης δεν θα μπορεί να επιβεβαιώσει καμία καταγραφή.
	\item  Δεν θα μπορεί να κάνει αναφορά ανακρίβειας για καταγραφή τιμής που αφορά
	άλλο πρατήριο εκτός του δικού του.
	\item  Δεν θα έχει πρόσβαση σε στοιχεία των εθελοντών πέραν αυτών που είναι
	δημοσίως διαθέσιμα.
\end{itemize}

\subsection{Δείκτες ποιότητας}


\begin{itemize}
	
	\item Εμπειρία χρήσης της πλατφόρμας (ευχρηστία, αποδοτικότητα, αποκρισιμότητα).
	\item Διαθεσιμότητα (με την έννοια του χρόνου) του συστήματος.
	\item Αμεσότητα απόκρισης από την εταιρία σε αναφορές ανακρίβειας.
	\item Πληρότητα, ακρίβεια και ευχρηστία των στατιστικών και των δεδομένων.
	\item Ποιότητα υποστήριξης από την εταιρία.
	\item Πλήθος, εγκυρότητα και συχνότητα καταγραφών για το πρατήριο.
	\item Συνολικό πλήθος χρηστών της εφαρμογής (συνεπώς και κοινού στο οποίο
	προβάλλεται το κατάστημα).
	
\end{itemize}

\section{Έκθεση απαιτήσεων χρηστών}
\begin{itemize}
	\item Λειτουργικές
	{
		\begin{itemize}
			\item Να βλέπει τις τιμές των ανταγωνιστών του.
			\item  Να ενημερώνει / αλλάζει / επιβεβαιώνει / καταγγέλλει τις τιμές για το
			πρατήριο του.
			\item  Να μπορεί να μπλοκάρει χρήστες ώστε να μην μπορούν να κάνουν
			καταγραφές για το πρατήριο του.
			\item  Να ορίζει τους ανταγωνιστές του (με απαρίθμηση, ή βάση κριτηρίων, πχ
			γεωγραφικών).
			\item  Να λαμβάνει ειδοποίηση όταν αλλάζουν οι τιμές των ανταγωνιστών του.
			\item  Να λαμβάνει ειδοποίηση για κάθε νέα καταγραφή (και άλλα σχετικά
			συμβάντα) για το πρατήριο του.
			\item  Να λαμβάνει αναλύσεις, δεδομένα, στατιστικά σχετικά με την
			ανταγωνιστικότητα της επιχείρησης του και τη βελτίωση αυτής.
			\item  Να προσθέτει πληροφορίες (πχ πρόσθετες παροχές) σχετικά με το πρατήριο
			του.
			\item  Να μπορεί να διαγραφεί από το παρατηρητήριο (και να αφαιρεθεί το πρατήριο
			του).
			\item  Να επαληθεύει την εγκυρότητα των πόντων προς εξαργύρωση ενός πελάτη.
		\end{itemize}
	}
	\item Μη λειτουργικές
	{
		\begin{itemize}
			\item Η συμμετοχή του στο παρατηρητήριο να μην απαιτεί συνεχή ενασχόληση.
			\item Να μην απαιτεί ιδιαίτερο εξοπλισμό ή/και λογισμικό για να λάβει μέρος.
			\item Το περιβάλλον χρήσης της πλατφόρμας να είναι χρηστικό, αποκρίσιμο και
			αποτελεσματικό.
			\item Να υπάρχει σεβασμός και διασφάλιση της ιδιωτικότητας των προσωπικών του
			δεδομένων.
			\item Η πλατφόρμα να είναι ασφαλής (πχ απέναντι σε διαρροές ή υποκλοπές
			δεδομένων, να μην μπορεί κάποιος τρίτος να αποκτήσει πρόσβαση στο
			λογαριασμό του καταστηματάρχη).
			\item Το σύστημα να είναι συνεχώς διαθέσιμο.
			\item Να προσφέρει ικανοποιητική υποστήριξη για τη χρήση του με τη μορφή
			οδηγών χρήσης ή εγχειριδίου αλλά και προσωποποιημένη, μέσω επικοινωνίας
			με κάποιον άνθρωπο.
		\end{itemize}
	}
\end{itemize}

\section{Αρχές του προτεινόμενου συστήματος}

\begin{itemize}
	\item Το παρατηρητήριο υλοποιείται από μία διαδικτυακή εφαρμογή.
	\item Τις καταγραφές τιμών τις καταχωρούν μόνο οι εθελοντές χρήστες.
	\item Κάθε καταγραφή μπορεί να επιβεβαιωθεί ως έγκυρη από άλλους χρήστες (όχι τον
	πρατηριούχο).
	\item Ο πρατηριούχος μπορεί να αναφέρει μία καταγραφή για το πρατήριο του ως μη
	έγκυρη. Τότε ενεργοποιείται ο μηχανισμός διερεύνησης και εξακρίβωσης του
	συστήματος.
	\item Οι πελάτες μπορούν να εξαργυρώνουν πόντους που έχουν συλλέξει από τις
	καταγραφές τους ως έκπτωση στις αγορές τους. Το παρατηρητήριο είναι υπεύθυνο για την επαλήθευση της εγκυρότητας των πόντων και την επικύρωση
	της συναλλαγής αυτών.
	\item 	Ο καταστηματάρχης λαμβάνει ειδοποίηση από την πλατφόρμα για κάθε νέα
	καταγραφή τιμής για το πρατήριο του.
\end{itemize}

\section{Περιορισμοί στο πλαίσιο του έργου}

Στο πλαίσιο του έργου, οι καταστηματάρχες επιβάλλουν δύο περιορισμούς: ο πρώτος
έχει να κάνει με την ασφαλή και σύμφωνη με τη νομοθεσία διαχείριση των δεδομένων
τους που παρέχουν στο παρατηρητήριο, ενώ ο δεύτερος αφορά το σύστημα συλλογής
και εξαργύρωσης των πόντων των πελατών, το οποίο πρέπει επίσης να πληροί υψηλές
απαιτήσεις ασφάλειας καθώς αν εκτεθεί μπορεί να χρησιμοποιηθεί για οικονομικό
όφελος (και οικονομική ζημία του πρατηριούχου).

\section{Παράρτημα: ακρωνύμια και συντομογραφίες}

Ν/Α

\end{document}
