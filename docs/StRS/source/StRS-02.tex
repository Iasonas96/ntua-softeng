\documentclass[a4paper,oneside, 12pt]{article}

\usepackage[margin=0.7in]{geometry}
\usepackage[cm-default]{fontspec}
\usepackage{xunicode}
\usepackage{xltxtra}
\usepackage{xgreek}
\usepackage{graphicx}
\usepackage{hyperref}

\graphicspath{{../UML/}{UML/}}

\setmainfont[Mapping=tex-text]{Linux Libertine O}

\title{Stakeholders Requirements Specification\\
\begin{flushleft}
	Εταιρία
\end{flushleft}}
\date{\vspace{-5ex}}

\begin{document}
\maketitle
\section{Εισαγωγή}
\subsection{Ταυτότητα-επιχειρησιακοί στόχοι}

Η εταιρία επιδιώκει την ανάπτυξη λογισμικού που θα καλύψει την ανάγκη των
καταναλωτών για μία εφαρμογή που θα καθιστά δυνατή την παρατήρηση τιμών.
Συγκεκριμένα, το παρατηρούμενο προϊόν θα είναι τα καύσιμα. Βασικός στόχος
αποτελεί η επίτευξη υψηλής επισκεψιμότητας, γεγονός που θα καθιστά εφικτό
το οικονομικό κέρδος της εταιρίας από την εισαγωγή διαφημίσεων.

\subsection{Περίγραμμα επιχειρησιακών λειτουργιών}

Αρχικά, η εταιρία είναι υπεύθυνη για την ανάπτυξη του λογισμικού της
εφαρμογής. Αφού αυτό αναπτυχθεί, οι αρμοδιότητές της περιορίζονται στην
αναβάθμιση και τη συντήρηση του κώδικα. Συγκεκριμένα, στην περίπτωση πχ
που βγει μία νέα έκδοση κάποιου από τα εργαλεία λογισμικού στα οποία
βασίζεται η λειτουργία της εφαρμογής, οφείλουν οι developers της εταιρίας να
μεριμνήσουν, ώστε να επιτευχθεί η συμβατότητα του κώδικα της εφαρμογής με
την νέα έκδοση του εργαλείου, εξασφαλίζοντας έτσι ότι η εφαρμογή δεν θα
σταματήσει να λειτουργεί φυσιολογικά. Αντίστοιχα, αν ένας χρήστης απευθυνθεί
για οποιοδήποτε λόγω στη διαχείριση πχ για τεχνική υποστήριξη ή για να
καταγγελθούν ψευδείς καταγραφές, πρέπει να αναλάβει υπάλληλος να
ικανοποιήσει το αίτημα του εκάστοτε χρήστη.

\section{Αναφορές-πηγές πληροφοριών}

Η εφαρμογή μας στηρίχθηκε στις ακόλουθες παλαιότερες εφαρμογές:

\begin{itemize}
	
	\item \href{http://www.fuelprices.gr/}{Παρατηρητήριο Τιμών Υγρών Καυσίμων }
	του Υπουργείου Ανάπτυξης και Ανταγωνιστικότητας
	
	\item \href{https://fuelgr.gr}{Εφαρμογή }
	ενημέρωσης για πρατήρια και τιμές πώλησης καυσίμων
	
	
\end{itemize}

\section{Διαχειριστικές απαιτήσεις επιχειρησιακού περιβάλλοντος}
\subsection{Επιχειρησιακό μοντέλο}
Η λειτουργία της εφαρμογής βασίζεται στο μοντέλο του πληθοπορισμού.
Αναλυτικότερα, το σύστημα στηρίζεται στην ύπαρξη εθελοντών. Συγκεκριμένα,
οι εθελοντές θα καταγράφουν τιμές για τα διάφορα πρατήρια, καθιστώντας έτσι
αυτή την πληροφορία προσιτή στον χρήστη. Ανάλογα με τη συμμετοχή του, ο
κάθε εθελοντής θα επιβραβεύεται από τους καταστηματάρχες με βάση ένα
σύστημα πόντων, τους οποίους θα μπορούν να εξαργυρώσουν μέσω εκπτώσεων
στα πρατήρια. Με τη σειρά της, η εταιρία θα δίνει τη δυνατότητα στους
καταστηματάρχες να ζητήσουν στατιστικά από αυτή, πληρώνοντας της το
κατάλληλο οικονομικό αντάλλαγμα. Τέλος, η συμμετοχή όλων θα δίνει τη
δυνατότητα στην εταιρία να εκμεταλλεύεται τον αριθμό των επισκέψεων
προκειμένου να έχει κέρδος από την εισαγωγή διαφημίσεων.

\subsection{Περιβάλλον διαχείρισης πληροφοριών}

Το περιβάλλον διαχείρισης πληροφοριών είναι μία σχεσιακή βάση δεδομένων. Οι
οντότητες σε αυτή τη βάση αποτελούνται από τα καταστήματα, τις
καταγραφές, τα προϊόντα, τους χρήστες και τους ιδιοκτήτες των πρατηρίων.

\section{Λειτουργικές απαιτήσεις επιχειρησιακού περιβάλλοντος}
\subsection{Επιχειρησιακές διαδικασίες}
Ο ρόλος της εταιρίας έχει διαχειριστικό χαρακτήρα. Αυτό σημαίνει ότι έχει
ευθύνη αφενός για τον ανά τακτά διαστήματα έλεγχο του κώδικα με σκοπό τη
συντήρηση και την εξέταση περιθωρίων επέκτασης και αναβάθμισης της εφαρμογής.
Αφετέρου, πρέπει να φροντίζει για την παροχή υπηρεσιών υποστήριξης προς
τους χρήστες. Σε αυτές περιλαμβάνεται η επικοινωνία με αυτούς, με σκοπό την
επίλυση αποριών σχετικά με τεχνικά ζητήματα, τον έλεγχο για αναφορές
σχετικά με λανθασμένες καταγραφές, καθώς και την επιβολή ποινών σε χρήστες
που επιδεικνύουν κακόβουλη συμπεριφορά (διαγραφή λογαριασμών κλπ).

\subsection{Περιορισμοί}

Η εταιρία περιορίζεται στο θέμα των προσωπικών δεδομένων. Δεν μπορεί να
ζητήσει από τους χρήστες κάτι πέρα τα στοιχεία που είναι απολύτως
απαραίτητα για την πιστοποίηση τους. Ακόμα και αν ζητηθούν τα ευαίσθητα
προσωπικά δεδομένα, που μπορούν να ταυτοποιήσουν τον χρήστη, δεν θα
κρατηθούν για παραπάνω από το ελάχιστο δυνατό διάστημα, όπως επιβάλλει το
γενικότερο νομοθετικό πλαίσιο με το οποίο οφείλει να εναρμονιστεί η εταιρία.
Επιπλέον, η εταιρία δεν επιτρέπεται να παρεμβαίνει μόνη της στις καταγραφές.
Αυτό σημαίνει πως δεν μπορεί να ελέγξει κατά πόσο είναι έγκυρη μία
καταγραφή, ούτε να αμφισβητήσει μία καταγγελία. Από την άποψη αυτή, ο
ρόλος των εργαζομένων που εκτελούν χρέη διαχειριστή είναι περισσότερο
επιτελικός, με την έννοια ότι δεν μπορούν τα μέλη της να τροποποιήσουν τα
στοιχεία της βάσης δεδομένων χωρίς να τους ζητηθεί.

\subsection{Δείκτες ποιότητας}
Δείκτες ποιότητας αποτελούν ποσότητες που δείχνουν ότι η εφαρμογή έχει
μεγάλη απήχηση στο κοινό στο οποίο απευθύνεται. Μερικά χαρακτηριστικά
μεγέθη είναι ο αριθμός των καταγραφών, το πλήθος των επισκέψεων, καθώς και
τα έσοδα από τις διαφημίσεις. Συγκεκριμένα, το πρώτο δείχνει το ενδιαφέρον
που υπάρχει από τη μεριά των εθελοντών ενώ τα άλλα δύο την επισκεψιμότητα
που παρατηρείται γενικότερα.

\section{Έκθεση απαιτήσεων χρηστών}
Καταρχάς, σε ότι αφορά τις λειτουργικές αρχές, κύρια απαίτηση από την μεριά
της εταιρίας είναι οι δραστηριότητες που αφορούν την εφαρμογή να είναι
επικερδείς, κάτι που πρέπει να υλοποιηθεί μέσω ύπαρξης διαφημίσεων και επί
πληρωμή παροχής στατιστικών δεδομένων στα συνεργαζόμενα πρατήρια
καυσίμων.
Στις μη λειτουργικές απαιτήσεις ανήκει η αποδοτική υλοποίηση του λογισμικού σε
ό,τι αφορά τους πόρους (ανθρώπινο δυναμικό). Επίσης το λογισμικό πρέπει να
είναι εύκολα συντηρήσιμο και ο κώδικας να είναι ευανάγνωστος. Τέλος πρέπει
να υπάρχει λεπτομερής τεκμηρίωση όλων των λειτουργιών του.

\section{Αρχές του προτεινόμενου συστήματος}

Η υπηρεσία υλοποιείται ως μία διαδικτυακή εφαρμογή. Οι εθελοντές
καταχωρούν τιμές που μπορούν να επιβεβαιωθούν ως έγκυρες από άλλους
εθελοντές αλλά όχι από τους εργαζόμενους της εταιρίας. Οι πρατηριούχοι
μπορούν να καταγγείλουν μία καταγραφή ως λανθασμένη, εφόσον αυτή αφορά
στο πρατήριο τους και μόνο σε αυτό. Σε αυτή την περίπτωση, αναλαμβάνει ο
διαχειριστής του συστήματος (που είναι εργαζόμενος της εταιρίας), που οφείλει
να ανταποκριθεί στο αίτημα του πρατηριούχου, διαγράφοντας την καταγραφή.

\section{Περιορισμοί στο πλαίσιο του έργου}

Καταρχάς, περιοριζόμαστε όσον αφορά στη γλώσσα στην οποία πρέπει να
λειτουργεί η ιστοσελίδα. Συγκεκριμένα, υποστηρίζονται μόνο τα ελληνικά.
Όσον αφορά στις δυνατότητες που έχουμε στα πλαίσια της εργασίας μας για
την ανάπτυξη της εφαρμογής, περιοριζόμαστε τόσο ως προς τον διαθέσιμο
χρόνο, όσο και ως προς τον αριθμό των ατόμων που εργάζονται και τη
χρηματοδότηση. Αναλυτικότερα, η εφαρμογή πρέπει να αναπτυχθεί στα πλαίσια
ενός πανεπιστημιακού μαθήματος, οπότε περιοριζόμαστε χρονικά από τη
διάρκεια του ακαδημαϊκού εξαμήνου. Επιπρόσθετα, η ομάδα μας αποτελείται
από πέντε άτομα, όπου και οι πέντε είναι πρώτη φορά που συμμετέχουμε σε
project που αφορά στην ανάπτυξη λογισμικού τέτοιας κλίμακας. Τέλος, η
εργασία μας δεν υποστηρίζεται οικονομικά από κανέναν χορηγό.

\section{Παράρτημα: ακρωνύμια και συντομογραφίες}

Ν/Α

\end{document}
